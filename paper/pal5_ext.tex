% Copyright 2019 the authors. All rights reserved.

% TODO:
% -

%\documentclass[modern]{aastex62}
\documentclass[twocolumn]{aastex62}
\usepackage{amsmath}

% typography
\setlength{\parindent}{1.\baselineskip}
\newcommand{\acronym}[1]{{\small{#1}}}
\newcommand{\package}[1]{\textsl{#1}}
\newcommand{\gaia}{\textsl{Gaia}}
\newcommand{\pans}{\textsl{Pan-STARRS}}
\newcommand{\DR}{\acronym{DR2}}
\newcommand{\msun}{\textrm{M}_\odot}
\newcommand{\kpc}{\textrm{kpc}}
\newcommand{\kms}{\ensuremath{\textrm{km}~\textrm{s}^{-1}}}
\newcommand{\bs}[1]{\boldsymbol{#1}}
\newcommand{\masyr}{\ensuremath{\textrm{mas}~\textrm{yr}^{-1}}}
\newcommand{\feh}{\ensuremath{[\textrm{Fe} / \textrm{H}]}}
\newcommand{\given}{\,|\,}

\newcommand{\sectionname}{Section}
\newcommand{\equationname}{Equation}
\renewcommand{\tablename}{Table}

\newcommand{\todo}[1]{{\color{red} TODO: #1}}

\newcommand{\changes}[1]{{\textbf{#1}}}
% \newcommand{\changes}[1]{{#1}}


% aastex parameters
% \received{not yet; THIS IS A DRAFT}
%\revised{not yet}
%\accepted{not yet}
% % Adds "Submitted to " the arguement.
%\submitjournal{ApJ}
\shorttitle{Cold streams}
\shortauthors{Pearson et al. }

%@arxiver{}

\begin{document}\sloppy\sloppypar\raggedbottom\frenchspacing % trust me
% All code used in this work and all results are available at
% \url{https://github.com/adrn/GD1-DR2}.
\title{Detecting Thin Stellar Streams in External Galaxies:\\ Resolved Stars \& Integrated Light}


 \author{Sarah Pearson}
 \affiliation{Center for Computational Astrophysics, Flatiron Institute, 162 5th Av., New York City, NY 10010, USA}
 \email{spearson@flatironinstitute.org}
 \correspondingauthor{Sarah Pearson}

\author{Tjitske Starkenburg}
\affiliation{Center for Computational Astrophysics, Flatiron Institute, 162 5th Av., New York City, NY 10010, USA}

\author{Kathryn V. Johnston}
\affiliation{Department of Astronomy, Columbia University, Mail Code 5246, 550 West 120th Street, New York, New York 10027, USA}

\author{Rodrigo A. Ibata}
\affiliation{Observatoire astronomique de Strasbourg, Université de Strasbourg, CNRS, UMR 7550, 11 rue de l’Université, F-67000 Strasbourg, France}

\begin{abstract}\noindent 
Stellar streams can be used to test the nature and behavior of dark matter. By investigating the morphology of thin stellar streams from globular clusters, the 
power spectrum of dark matter substructure can potentially be uncovered, as well as the shape of dark matter halos. It is therefore crucial to extend our searches for thin globular cluster streams to other galaxies than the Milky Way. In this paper, we investigate the current and future prospects of detecting globular cluster streams in external galaxies in both resolved stars (e.g. with WFIRST) and using integrated light (e.g. with HSC and LSST). In particular, we inject mock-streams to real data from the PAndAS M31 survey, and create simulated M31 backgrounds mimicking what WFIRST will observe in M31. Additionally, we estimate the distance limit to which globular cluster streams will be observable in resolved stars and integrated light. Our results demonstrate that  WFIRST should easily detect a Pal 5 like stream in resolved stars in external galaxies out to distances of xx MPC. With integrated light, we conclude that thin streams are observable out to 600 Mpc, depending on the stream properties and the properties of the host galaxy. We conclude that WFIRST will discover a vast number of thin stellar streams with resolved stars in external galaxies. We emphasize that if these external galaxies do not host spiral arms, bars and molecular clouds, gaps in thin streams in such galaxies are ideal laboratories in which to search for evidence of interactions between dark matter subhalos and stellar streams. Statistical samples of thin stellar streams can further help constrain orbit distributions and potentially triaxiality of external halos. 

\end{abstract}
\keywords{{\bf Key words:} dark matter — Galaxy: halo — Galaxy: structure — Galaxy: kinematics
and dynamics — globular clusters: individual (Palomar 5): }% {\bf Methods:} }

%----------------------------------------------------Intro ----------------------------------------------------------------
\section{Introduction} \label{sec:intro}
Stellar streams form when a gravitationally bound ensemble of stars tidally tears apart, due to an underlying galactic potential. To date, we have observed a multitude of stellar streams in our own Galaxy, emerging as leading and trailing tidal tails from both open clusters (e.g. \citealt{roser19}) and globular clusters (e.g. GD1: \citealt{grillmair06}, Palomar 5: \citealt{oden01}), as well as dwarf galaxies (e.g. Sagittarius: \citealt{ibata01}, Orphan: \citealt{belokurov06}). Following the release of Gaia DR, more than 60 stellar stream candidates have been suggested in the Milky Way alone (e.g. \citealt{ibata19}). 
Several stellar streams have additionally been discovered in external galaxies (e.g. \citealt{ibata00}, \citealt{delgado10}). Based on the widths and surface brightnesses of the stellar streams in external galaxies, these streams are likely relics from tidally disrupted dwarf galaxies (e.g. \citealt{delgado12}).

%GCs as potential probes
Since the discovery of stellar streams, several studies have proposed to use the observed properties of streams to measure the mass distribution in our Galaxy, including its dark matter (e.g. \citealt{johnston99}, \citealt{koposov10}, \citealt{law10}, \citealt{bovy16}). Stellar streams from globular clusters (GCs) are of particular interest, as they are dynamically cold (i.e. the internal kinematics of globular clusters are much smaller than the clusters' orbital velocities around their host galaxies). As a consequence, the streams from GCs phase-mix slowly and leave behind stars moving coherently in phase-space along thin leading and trailing arms for several gigayears (Gyr) that can be dense enough to be detectable with todays surveys. %This makes thin GC stellar streams sensitive probes of the underlying mass distribution of their hosts. 
Moreover, because they are so cold, these streams are particularly sensitive to any deviations to smooth, symmetric potentials - and hence particularly useful for probing dark matter distributions.

%GC streams as LCDM probes
While studies to measure the potential of the Galaxy have typically relied on multiple dimensions of data, including kinematics, there are some specific examples where the morphology of thin streams are alone informative. For example, the $\Lambda$-cold dark matter ($\Lambda$CDM) paradigm predicts a specific distribution and mass range of dark matter subhalos in our Galaxy (see e.g. \citealt{diemand08}, \citealt{bovy17}, \citealt{bonaca19}). Additionally, \citet{ibata02} and \citet{johnston02} showed that the interaction between dark matter subhalos can leave behind signatures in the structure of stellar streams. Density fluctuations in GC stellar streams, can therefore, in principle, provide indirect evidence of interactions with dark matter substructure, and serve as a test of $\Lambda$CDM (e.g. \citealt{yoon11}, \citealt{erkal16}, \citealt{bovy17}, \citealt{bonaca19}).

%GC streams to probe orbit structure
In addition, the morphology of GC stellar streams provide broad constraints on  dark matter halo shapes, as only certain orbits in triaxial matter distributions allow thin, long streams to exist (\citealt{pearson15}). In particular, thin, long stream should only be detectable on regular or resonantly trapped orbits, so their presence and location can provide a map of these regions in the orbit structure of a potential (\citealt{pearson15}, \citealt{price16}, Yavetz et al., {\it in prep.}).

The fact that useful information can be extracted from the morphology of thin GC streams alone opens up the exciting possibility of applying some of the intuition built for streams around our own Milky Way to other galaxies.  GC streams, however, have lower masses and are thinner than streams from dwarf galaxies, and are therefore harder to detect in external galaxies. Interestingly, \citet{abraham18} reported a detection of the ``Maybe Stream" with HST, which they suggest could be a GC stream 20 Mpc away. 

%Therefore, finding cold stellar streams in external galaxies will help us map the orbit structures and shapes of the potentials, as t%This paper

In this paper, we investigate the future prospect of observing thin, globular cluster streams in external galaxies both through resolved stars and integrated light. 
%In particular, we focus our attention on the well-studied stream, Palomar 5. 
%, which was first discovered in SDSS (Oden03), and which has since been mapped in great detail by (\citealt{ibata16}). 
Specifically, we ask whether globular cluster streams will be observable in resolved stars with upcoming telescopes such as WFIRST (\citealt{spergel13}), or in integrated light with current telescopes such as the Hyper Suprime-Cam (HSC: \citealt{miyazaki12}) and future telescopes such as LSST (\citealt{ivezi08}). % In particular, we explore the observability of globular cluster streams at various locations in their host galaxies, and we ask how far away the host galaxies can be for the thin streams to be detectable.

%Organization of paper 
The paper is organized as follows: in Section \ref{sec:coldstreams}, we describe the properties of globular cluster streams and how we create mock-streams to test whether they are observable. In particular we describe Palomar 5 (Pal 5) which we use as our fiducial model (Section \ref{sec:pal5}), we describe how we populate our streams with stars (Section \ref{sec:lum}), and we calculate widths and lengths of mock streams at various galactocentric radii (Section \ref{sec:length}). In Section \ref{sec:results}, we present the results on detecting streams in resolved stars in the Andromeda galaxy (M31) (Section \ref{sec:resolved}), in other external galaxies (Section \ref{sec:resother}) and in integrated light (Section \ref{sec:integrated}). We discuss the implications of our results in Section \ref{sec:discussion} and conclude in Section \ref{sec:conclusion}.
%

\section{Cold globular cluster stellar streams}
\label{sec:coldstreams}
Our goal is to estimate the observability of GC stellar streams in external galaxies. In this Section, we describe the framework we use to create mock stellar streams.  We chose the Milky Way's stellar stream, Pal 5,  as our reference stream, because the progenitor cluster of Pal 5's stream has been observed, which is only the case for very few known streams. Knowing the progenitor, enables us to determine the properties of the overall system more precisely (such as age, mass, metallicity, orbit etc.). 


\begin{figure*}
\centerline{\includegraphics[width=\textwidth]{FIG1_CFHT_WFIRST_IMF3000.pdf}}
\caption{%{\bf Cold streams: what will  observe.} \\
{\bf Top Left:} CHFT $g, i$ color-magnitude diagram (CMD) of a Pal 5-like cluster from the PARSEC system moved to the distance of Pal 5 in the Milky Way (d = 23 kpc). We use the same values for the cluster as \citet{ibata17}: the age of the cluster is set to 11.5 Gyr and [Fe/H] $= -1.3$. The three horizontal lines show faintest magnitude, SDSS and CHFT can observe for the cluster at Pal 5's current distance in the Galaxy ($d = 23$ kpc), and the faintest magnitude PAndAS observes in M31 ($d = 785$ kpc). {\bf Top Middle}: The normalized luminosity function for the PARSEC Pal 5-like cluster. We have normalized the luminosity function such that the amount of stars between $20 < g < 23 = 3000 \pm 100$ (Bonaca et al., {\it in prep.}). %We computed the luminosity function by sampling a full power law initial mass function (IMF: $m^{-0.5}$) between $0.01 - 120 ~\msun$. 
{\bf Top Right:} The cumulative number of stars in a Pal 5-like stream (solid) and a 10 times more massive Pal 5-like stream (dashed) for a given limiting $g$-mag. The vertical lines show the limiting magnitude of SDSS ($g < 22.5$) and PAndAS ($g < 25.5$) at the distance of M31 (i.e. shifted by 7.66 magnitudes from Pal 5's current location), and the limiting magnitudes of SDSS ($g < 22.5$) and CHFT ($g < 24$) at the distance of Pal 5 in the Milky Way. We indicate the amount of stars PAndAS should be able to observe for the Pal 5-like and 10 $\times$  Pal 5-like stream in M31 (see n$^*$). 
%We note that PAndAS should be able to observe $\sim$59 stream stars at the distance of M31 for a Pal 5-like cluster and $\sim$375 stream stars for a stream emerging from a cluster with 10 times the mass of Pal 5. 
{\bf Bottom Left:} This row shows the same as top row but for WFIRST bands ($R062$ and $Z062$) and limiting magnitudes (Z062 $<$ 27.15). {\bf Bottom middle:} To compute the WFIRST luminosity function, we sampled the exact same initial masses from the WFIRST isochrone downloaded from the PARSEC system as for the $g$-band above. This leads to a different number of stars between the magnitude of 20 - 23, which is consistent with what WFIRST should observe for a Pal 5-like stream. {\bf Bottom right:} We note that WFIRST should be able to observe $\sim$245 stream stars at the distance of M31 for a Pal 5-like cluster and $\sim$2467 stream stars for a stream emerging from a cluster with 10 times the mass of Pal 5. All of these stars should be on the red giant branch and later evolutionary stages (see left panel).}
\label{fig:iso_cfht}
\end{figure*}



\subsection{Pal 5 data}
\label{sec:pal5}
 \citet{ibata16} presented photometric data of Pal 5 taken with the MegaCam instrument at the 3.6m Canada-France-Hawaii Telescope (CFHT) during 2006-2008. The CFHT $g, r$ bands provide data down to $g_0$ = 24 with good precision around the cluster main-sequence turnoff. In particular, their sample of stars with 20 $< g <$ 23 has a completeness of 80\% (see \citealt{ibata16} and \citealt{ibata17} for more details). From \citet{ibata16} figure 7, we estimate that there are 1767 stars in the stream between 20 $< g <$ 23 over a length of 20 $\deg$  and a width of 0.14 $\deg$ if we exclude the cluster stars, which can be used to estimate how many stream stars there are in Pal 5's stellar stream.
 
Recently, Bonaca et al. ({\it in prep.}) obtained new DECaLS data of Pal 5, and found that there were $3000 \pm 150$ stars in the Pal 5 stream between 20 $< g <$ 23 excluding the cluster and after subtracting the background. Both Bonaca et al. ({\it in prep.}) and  \citet{ibata16} reach a similar limiting magnitude of ($g < 24$), and the discrepancy in the number of stars  between 20 $< g <$ 23 for the two data sets is likely due to a better star galaxy separation in DECaLS. Throughout the paper, we use $3000 \pm 150$ stars between 20 $< g <$ 23 to estimate how many stream stars there are in Pal 5's stellar stream at an given magnitude (see section below).
%We use this sample to estimate how many stream stars there are in Pal 5's stellar stream at an given magnitude (see section below). From \citet{ibata16} figure 7, we estimate that there are 1767 stars in the stream between 20 $< g <$ 23 over a length of 20 $\deg$  and a width of 0.14 $\deg$ if we exclude the cluster stars. \todo{Ibata+ 2016 did not remove underlying background population of Galactic (and Sagittarius stream) stars that contaminate the Palomar 5 sample. In Ana Bonaca's new DECaLS data, we see $\sim$ 3000 stars between 20 $< g <$ 23 after subtracting the background. I might update the paper to use those numbers instead.}  

\subsection{Isochrones \& luminosity functions of streams}
\label{sec:lum}
To construct mock streams, we first download isochrones for a Pal 5-like cluster from the PARSEC evolutionary tracks (\citealt{bressan12}). We use the same values for the Pal 5-like cluster as \citet{ibata17}: the age of the cluster is set to 11.5 Gyr and the metallicity is fixed as [Fe/H] $= -1.3$. We download the isochrone for both CFHT bands and WFIRST bands and shift the isochrones to Pal 5's current distance in the Milky Way (d = 23 kpc, $d_{mod} = 16.86$, see left panel of Figure \ref{fig:iso_cfht}). 

Subsequently, we compute a luminosity function, by sampling masses ($m = 0.01 - 120 ~\msun$) from a power law initial mass function (IMF: $dN/dm \propto m^{-0.5}$), and use the isochrone tables for the age = 11.5 Gyr and [Fe/H] $= -1.3$ to infer the stars' magnitudes. 

To determine how many stars we can observe in a Pal 5-like stream at a given limiting magnitude, we normalize the luminosity function from CFHT such that there are  $3000 \pm 150$ stars (Bonaca et al., {\it in prep.}) with $20 < g < 23$ (see Figure \ref{fig:iso_cfht}, top, middle panel). We then compute the cumulative number of stars at a given limiting magnitude (see Figure \ref{fig:iso_cfht}, top right panel, solid line).
%3000 +/- 150 given poisson uncertainty in the background and stream

Globular clusters have a spread in masses (\citealt{harris96}), and some are more massive than the inferred initial mass of Pal 5 ($\sim 50,000 ~\msun$: \citealt{ibata17}). We therefore repeat the above exercise for a ten times more massive stream than Pal 5 (dashed line). The dashed line was computed by resampling the IMF such that there are are $\sim$ 30,000 stars between $20 < g < 23$ (ten times more than for the observed Pal 5-stream). 

In the top, right panel of Figure \ref{fig:iso_cfht}, we note that the PAndAS survey should detect $\sim 59 \pm 10$ stars in a Pal 5-like stream at the distance of M31 ($d_{mod} = 24.46$, see vertical line) and $\sim 610 \pm 100$ stars for a stream that is ten times more massive than Pal 5. The scatter arises due to the IMF sampler and we have sampled both IMFs 100 times to estimate the spread. All of the observable stars are at brighter magnitudes than the main sequence turnoff (see left panel of Figure \ref{fig:iso_cfht}) and are in the red giant branch (RGB) or more evolved stellar evolutionary stages. The four vertical lines show the limiting magnitudes of various surveys at the distance of Pal 5's current location and if it were residing in M31.

In the bottom row of Figure \ref{fig:iso_cfht}, we show the same panels as in the top row, but now using downloaded PARSEC Pal 5-like isochrones in WFIRST $R062$ and $Z062$ bands. We used the same properties of the Pal 5-like cluster as described above (i.e. 11.5 Gyr and [Fe/H] $= -1.3$). 

To estimate how many stars WFIRST should be able to observe in a Pal 5-like stream, we normalized the luminosity function by sampling the exact same initial masses in the dowloaded WFIRST isochrone as for the $g$-band CFHT masses. This provides us with the number of stars WFIRST will observe at a given magnitude. 

At the distance of M31, WFIRST will be able to detect $\sim$ 245$\pm 10$ stars in a Pal 5-like stream and $\sim$  2467$\pm 100$ stars for a stream that is ten times more massive than Pal 5 (Figure \ref{fig:iso_cfht}, bottom right panel). We explore the observability of resolved stars in GC streams in M31 with the PAndAS survey (\citealt{mcconnachie09}) and with WFIRST in Section \ref{sec:resolved}. Note that if the streams are located more than 11.5 Mpc away, even the brightest giants will not be resolved with WFIRST's guest observer program, as the limiting magnitude is $Z062 <$  27.15. 


%-----------------------------------Table-----------------------------------
\begin{table*}
\centering
\caption{Properties of mock streams in M31-like halo}
\label{tab:Pal5}
\begin{tabular}{lccc}
\hline
 & {\bf R$_{GC}$ = 15 kpc }&  {\bf R$_{GC}$ = 35 kpc} &  {\bf R$_{GC}$ = 55 kpc} \\ 
 \hline
 %&[kpc]&[kpc]&[kpc]&[]& [arcsec$^2$]  & mag/[arcsec$^2$]  \\
{\bf Pal 5-like mass} & &&\\
$l$ [kpc] & 7.8 & 10.4 &  12.0  \\
$w$ [kpc] & 0.053 & 0.094  & 0.127  \\
$Area$ [kpc$^2$] & 0.414 & 0.973 &1.53 \\
%$\sqrt{N_{\rm PAndAS, background}}$\footnote{Number of stars in background for an equal area as stream covers}& .& .&. \\
%$\sqrt{N_{\rm WFIRST, background}}^{\rm a}$& .& .&. \\
%$\frac{N_{\rm stream,stars}}{\sqrt{N_{\rm PAndAS}}}$& .& .&. \\
%$\frac{N_{\rm stream,stars}}{\sqrt{N_{\rm WFIRST}}}$& .& .&. \\
$\mu$ [mag/arcsec$^2$] ($Z062$-band) &32.1 &33.0  &33.7  \\
\hline
{\bf 10 $\times$ Pal 5-like mass} & &&\\ 
$l$ [kpc] &16.8 & 22.3 & 25.9 \\
$w$ [kpc]& 0.115 & 0.202 & 0.273  \\
$Area$ [kpc$^2$] &  1.93& 4.51   & 7.08  \\
%$\sqrt{N_{\rm PAndAS, background}}^{\rm a}$& .& .&. \\
%$\sqrt{N_{\rm WFIRST, background}}^{\rm a}$& .& .&. \\
%$\frac{N_{\rm stream,stars}}{\sqrt{N_{\rm PAndAS}}}$& .& .&. \\
%$\frac{N_{\rm stream,stars}}{\sqrt{N_{\rm WFIRST}}}$& .& .&. \\
$\mu$ [mag/arcsec$^2$] ($Z062$-band) &31.2 &32.1  &32.6  \\

\hline 
%%%%%%%%%%%%%%%%%%%%%%%%%%%%%%%%%%%%%%%%%%
\end{tabular}
\end{table*}



\subsection{Length \& width of streams}
\label{sec:length}
In the previous section, we found the amount of observable stars in GC streams at the distance of M31 for the limiting magnitudes of PAndAS and WFIRST. In this Section, we describe how we populate mock streams with the given number of stars. In particular, we compute the widths and lengths of mock GC streams on circular orbits, taking Pal 5 as our starting point. 

To place a Pal 5-like stream in the M31 halo, % we assume a flat rotation curve (constant $v_{circ}$) and keep the age of Pal 5 fixed. Hence Pal 5's dynamical time to form a stream remains unchanged. 
we scale the mock streams from the initial length and width of Palomar 5 in our Galaxy: length = 8.5 kpc, width = 58 pc (\citealt{ibata16}). For simplicity, we assume a circular orbit with a distance from the center of the Galaxy of R$_{GC}$ = 15 kpc, which is similar to the average of Pal 5's actual apocentric and pericentric distance for its rather elliptical orbit (see e.g. \citealt{erkal17}). 

We follow \citet{johnston98} and \citet{johnston01}, to compute the width and lengths of the streams at three different galactocentric radii: $R_{GC}$ = 15 , 35, 55 kpc in M31's halo. We compute lengths and widths for streams with two different stream masses: 1) a Pal 5-like stream mass 2) a ten times more massive Pal 5-like stream. Note that we do not actually assume a mass for the cluster, but scale stream properties based on the observed properties of Pal 5 in the Milky Way today (i.e. the length, width presented in \citealt{ibata16} and the number of stars presented in Bonaca et al., {\it in prep.}).  

Following \citet{johnston01} equation 8, we can express the width, $w$, of the streams as:

\begin{equation}
w = R_{GC} \left[\frac{m}{M(R_{GC})}\right]^{1/3} = R_{GC} \left[\frac{m G}{v_c^2 R_{GC}}\right]^{1/3} 
\end{equation}
where $R_{GC}$ is the radius of the orbit of the globular cluster around its host (which is normally expressed as $R_p$, but since we have a circular orbit, this remains constant over the entire orbit), $m$ is the mass of the cluster, $M(R_{GC})$ is the enclosed mass of the host within the stream's orbit, $v_c$ is the circular velocity at radius, $R_{GC}$, and $G$ is the gravitational constant. We assume a flat rotation curve and therefore constant  $v_c$, which is a valid assumption at $R_{GC} > 15$ kpc in M31 (e.g. \citealt{chemin09}). Thus the width of a GC stream scales as:

\begin{equation}
\label{eq:w}
w \propto R_{GC}^{2/3} \frac{m^{1/3}}{v_c^{2/3}}
\end{equation}

We scale the width of Pal 5 based on each parameter in Equation \ref{eq:w} separately. First, we update the width of the stream reflecting, that the circular velocity in M31, $v_{c, M31}= 250 ~\kms$  (\citealt{chemin09}) as opposed to  $v_{c, MW}= 220 ~\kms$. Hence, we correct the width to be narrower by a factor of $v_{corr} =  \left(\frac{v_{c,M31}}{v_{c,MW}}\right)^{-2/3} \sim 0.92$. 

Subsequently, we update the width of the stream based on the location in M31's halo. Thus, as the stream is placed at larger radii, we enlarge the width by a factor of  $R_{corr} = \left(\frac{R_{GC}}{R_{GC,Pal5}}\right)^{2/3}$. Lastly, we scale the mass up by a factor of ten and therefore make the stream a factor of $m_{corr} = \left(\frac{10m}{m}\right)^{1/3}= 2.154$ wider for the more massive streams. Recall that we do not assume anything about the Pal 5 cluster's initial or present day mass. 


Following \citet{johnston01} equation 5, we can express the angular length, $\Psi$, of the stream as:
\begin{equation}
\Psi = 4  \left[\frac{m}{M(R_{GC})}\right]^{1/3}  \frac{2 \pi t}{T_{\psi}} \propto \left[\frac{m}{M(R_{GC})}\right]^{1/3}  \frac{t}{T_{\psi}}
\end{equation}
where $t$ is the dynamical age of the stream, and $T_{\psi}$ is the azimuthal period of the cluster around its host galaxy. We can re-write the angular length as:
\begin{equation}
%L \propto  R_{GC} \left(\frac{m }{v_c^2 R_{GC}}\right)^{1/3}  \frac{v_c}{R_{GC}} = \frac{m^{1/3}}{R_{GC}^{1/3}} v_c^{1/3}
\Psi \propto \left[\frac{m }{v_c^2 R_{GC}}\right]^{1/3}  \frac{v_c t }{R_{GC}} = \frac{m^{1/3}}{R_{GC}^{4/3}} v_c^{1/3}t
\end{equation}
The physical length of the stream can thus be expressed as:
\begin{equation}
\label{eq:l}
%L \propto  R_{GC} \left(\frac{m }{v_c^2 R_{GC}}\right)^{1/3}  \frac{v_c}{R_{GC}} = \frac{m^{1/3}}{R_{GC}^{1/3}} v_c^{1/3}
L \propto R_{GC} \frac{m^{1/3}}{R_{GC}^{4/3}} v_c^{1/3}t = \frac{m^{1/3}}{R_{GC}^{1/3}} v_c^{1/3}t
\end{equation}

We scale the length of Pal 5 based on each parameter in Equation \ref{eq:l} separately. We contrast streams all of the same age (i.e.  t = constant). We first make the stream longer by a factor of $(\frac{v_{c,M31}}{v_{c,MW}})^{1/3} = 1.04$. Subsequently, we update the length of the stream based on the location in M31's halo. Thus, as the stream is placed at larger radii, we shorten the length by a factor of  $R_{corr} = \left(\frac{R_{GC}}{R_{GC,Pal5}}\right)^{-1/3}$, reflecting the fact that these streams will have completed less orbits around their hosts. %(\todo{this is the same as before as I first did $R_{corr} = \left(\frac{R_p}{R_{p,Pal5}}\right)^{2/3}$ and then $ L_{corr} = \frac{R_{p,pal5}}{R_{p,new}}$  }). 
Additionally, we scale the length of the stream by a factor of $m_{corr} = 10^{1/3}$ for the more massive stream. 

To summarize, we have the framework to construct six different mock stellar streams of various lengths and widths scaled from Pal 5's initial properties of $L = 8.5$ kpc and $w = 58$ pc in the Milky Way (see Table \ref{tab:Pal5}). We populate the mock streams of various widths and lengths on great circles with radii of $R_{GC} =$ 15, 35, 55 kpc, and with the number of stars obtained in Section \ref{sec:lum}. Note that there are several ways we could have constructed mock streams, and the mock streams presented in this work is meant to illustrate a range of possible streams. In reality, the stellar streams could, of course, have a slightly different range of orbital properties, lengths, widths and number of stars.




%
%\begin{figure*}
%\centerline{\includegraphics[width=\textwidth]{WFIRST.pdf}}
%\caption{{\bf Cold streams: what can WFIRST observe.} Same as Figure \ref{fig:iso_cfht}, but for WFIRST bands and limiting magnitudes. {\bf Left:} WFIRST R062-band and Z062-band CMD of a Pal 5-like cluster downloaded from the PARSEC system using the same values for the cluster as \citet{ibata17}: the age of the cluster is set to 11.5 Gyr and [Fe/H] $= -1.3$. The two horizontal lines show faintest magnitude on the CMD that WFIRST can observe at the distance of M31 (d = 770 kpc) and at the current distance of Pal 5 (d = 23 kpc) in the Milky Way. {\bf Middle}: The normalized luminosity function for the PARSEC Pal 5-like cluster using the WFIRST Z062-band magnitudes. \todo{explain how normalization in g band translates to the WFIRST luminosity function}. {\bf Right:} The cumulative number of stars in a Pal 5-like stream (solid) and a 10 times more massive Pal 5-like stream (dashed) for a given limiting WFIRST Z062-band mag. The vertical lines show the limiting magnitude of WFIRST's guest observer program (Z062 $< 27.15$) at the distance of M31 (d = 770 kpc) and at Pal 5's current position (d = 23 kpc).  We note that WFIRST should be able to observe $\sim$100 stream stars at the distance of M31 for a Pal 5-like cluster and $\sim$1000 stream stars for a stream emerging from a cluster with 10 times the mass of Pal 5. All of these stars should be on the red giant branch and in later evolutionary stages \todo{would there really be this many RGB stars in the stream? Seems like a lot. }.}
%\label{fig:iso_wfirst}
%\end{figure*}

\begin{figure*}
\centerline{\includegraphics[width=\textwidth]{Fig2_M31_Pandas_mag255_IMF3000.png}}
\caption{{\bf Cold streams in M31 - with PAndAS ($g < 25.5$)}. In this plot we demonstrate the observability of a Pal 5-like stream and a 10 times more massive Pal 5-like stream for the PAndAS survey of the M31 halo. Each column shows the PAndAS data at three different M31 galactocentric radii ($R_{GC}$ = 15  kpc, $R_{GC}$ = 35 kpc and $R_{GC}$ = 55 kpc). In the first row we show all stars observed in the PAndAS fields, in the second row we show all stars observed in PAndAS with a metallicity [Fe/H] $< 0$, and in the third row we show all stars with [Fe/H] $< -1$. In each panel, we have injected two streams: a Pal 5-like stream and a Pal 5-like stream with 10 times the mass of Pal 5. Recall that the Pal 5-like stream has [Fe/H] $= -1.3$, and should therefore be visible in all three rows. 
We determine the number of stars in the two streams by summing up the cumulative number of stars in the streams at the limiting magnitude of the PAndAS survey ($g < 25.5$) at the distance of M31 (see Figure \ref{fig:iso_cfht}, right panel). There are are $\sim 610$ stars in the more massive stream and only $\sim 59$ stars in Pal 5-like stream. At each $R_{GC}$, we have updated the width and lengths of the streams based on the tidal field they experience at these distances. Additionally, the 10 times more massive stream is wider and longer by a factor of $10^{1/3}$ (see Section \ref{sec:length}). We note that hints of the Pal 5-like stream is visible in the lower right panels, and that the 10 times more massive stream becomes apparent further out in the halo when we apply metallicity cuts (see middle and right panels in row two and three). \todo{Att Rodrigo: We can show the homogeneous fields with i $<$ 24 instead of 25.5. Do you think I should do that?}  }
\label{fig:M31_pandas}
\end{figure*}




%\begin{figure}
%\centerline{\includegraphics[width=\columnwidth]{M31_Pandas_mag24.pdf}}
%\caption{{\bf Cold streams in M31 - with PANDAS ($i < 24$)}. Same as Figure \ref{fig:M31_pandas}, but using a limiting magnitude of  $i <$ 24 to ensure homogeneity between the fields. At this limiting magnitude, there are 100 stars in the ten times more massive stream and only 10 stars in the Pal 5-like stream. The length and width of streams are updated based on their location in galaxy and the mass of stream. \todo{We should probably just decide to use this or fig \ref{fig:M31_pandas}.}}
%\label{fig:M31_pandas24}
%\end{figure}




\section{Results}\label{sec:results}
In this Section, we present results on the detectability of GC stellar streams for resolved stars in M31 (Section \ref{sec:resolved}), for resolved stars in other external galaxies (Section \ref{sec:resother}), and by using integrated light (Section \ref{sec:integrated}). 
 

\subsection{Cold streams in M31: resolved stars}
\label{sec:resolved}
We first explore whether GC streams should be observable in M31 given the PAndAS limiting magnitudes (subsection \ref{sec:PANDAS}), and test the same scenario given the WFIRST limiting magnitudes and bands (subsection \ref{sec:WFIRST}). As a first step, we define observable as meaning ``easily apparent as an over density by eye".%Additionally, we explore what we can learn from color information and with lower metallicity streams (subsection \ref{sec:color}).


\subsubsection{PAndAS}
\label{sec:PANDAS}
The PAndAS survey (e.g. \citealt{mcconnachie09}) has mapped a total area 400 square degrees surrounding M31 using the 1-square-degree field-of-view MegaPrime/MegaCam camera on the 3.6m Canada-France-Hawaii Telescope. It surveyed in $g, i$-bands to depths of $g$ = 26.5, $i$ = 25.5, resolving individual stars with a signal-to-noise ratio of 10. The PAndAS team derive photometric metallicities for all stars by assuming that the width of the red giant branch (RGB) can be interpreted as the spread in metallicity within a galaxy (see e.g. \citealt{crno14}).

The columns in Figure \ref{fig:M31_pandas} show 10 $\times$ 10 kpc regions from the PAndAS survey of M31's halo at various galactocentric radii ($R_{GC}$ = 15  kpc, $R_{GC}$ = 35 kpc and $R_{GC}$ = 55 kpc), and each row shows the same areas after we have applied various metallicity cuts: [Fe/H] = all,  [Fe/H] = $>$ 0, and  [Fe/H] = $> -1 $, respectively. Recall that the Pal 5 cluster metallicity is  [Fe/H] $= -1.3$, and that all Pal 5 stars will therefore be visible for these metallicity cuts. 

In each panel, we have injected a Pal 5 mock stellar stream and mock stellar stream with properties scaled to have 10 times the mass of Pal 5. We have populated the streams with stars using the PAndAS limiting magnitude calculated in Section \ref{sec:lum} (59 and 610 stars for the less and more massive stream, respectively).  The streams' lengths and widths have been calculated based on the streams' distances from M31's galactic center. Additionally, the more massive stream is longer and wider by a factor of $10^{1/3}$ (see Section \ref{sec:length}). 

Hints of the Pal 5-like stream are visible in the lower right panels of Figure \ref{fig:M31_pandas} after we have applied metallicity cuts. The 10 times more massive stream is barely apparent by eye in the first row, but when we apply metallicity cuts (see row two and three), the more massive stream is easily detectable \footnote{Note that we have located the mock streams at the same positions in both Figure \ref{fig:M31_pandas} and \ref{fig:M31_wfirst}, in case the reader is curious to where the Pal 5-like streams are located in Figure  \ref{fig:M31_pandas}.}. The PAndAS team have not reported the detection of massive GC streams, thus M31 might not host GC streams that are much more massive than Pal 5's stream, which is the case for the MW also. However, it should be feasible to find a 10 times more massive Pal 5-like stream in the PAndAS data, should such a stream exist.

\begin{figure*}
\centerline{\includegraphics[width=\textwidth]{Fig3_M31_WFIRST_mag2715.png}}
\caption{{\bf Cold streams in M31 - with WFIRST}.  This Figure shows the same panels as Figure \ref{fig:M31_pandas}, we have now, however, determine the number of stars in each mock stream by summing up the cumulative number of stars in the streams at the limiting magnitude of WFIRST ($Z068 < 27.15$) at the distance of M31 (see Figure \ref{fig:iso_cfht}, botoom, right panel). Thus, there are 1519 stars in the more massive stream and 148 stars in Pal 5-like stream. At each $R_{GC}$, we have updated the width and lengths of the streams based on the tidal field they experience at these distances (see Section \ref{sec:length}). Additionally, we have updated the number of stars in each field such to illustrate what WFIRST will observe in M31 given WFIRST's  deeper limiting magnitude (see details in Section \ref{sec:WFIRST}).
We note that both the Pal 5-like at the ten times more massive stream are visible in most panels.}
\label{fig:M31_wfirst}
\end{figure*}

\subsubsection{WFIRST}
\label{sec:WFIRST}
NASA's Wide Field InfraRed Survey Telescope (WFIRST) is planned to launch in mid-2020s. The space telescope has a 2.4 meter primary mirror and its wide field instrument will have a field of view that is 100 times greater than the Hubble infrared instrument and has a pixel size of 0.11$\arcsec$. Thus, WFIRST is particularly useful for studying resolved stellar populations over large areas. 

To test whether a Pal 5-like stream will be detectable with WFIRST, we first simulate M31's halo for WFIRST bands and limiting magnitudes. We do this by populating the 10 $\times$ 10 kpc PAndAS panels from Figure \ref{fig:M31_pandas} ($R_{GC}$ = 15, 35, 55 kpc) with additional stars to mimic the densities and magnitudes WFIRST will observe. To do this, we first download a grid of isochrones, 16 in total, from the PARSEC system with a metallicity range of [Fe/H] = $-2.5$ - 0.5 in increments of [Fe/H] = 0.5, and an age range of age = 8.5 - 11.5 Gyr in increments of 1 Gyr. Subsequently, we fit all stars in the color-magnitude diagrams of the three 10 $\times$ 10 kpc  PAndAS M31 panels to these isochrones at the distance of M31, and at two distances representative of the Milky Way inner and outer halo to fit foreground stars. Using the number of stars fitted to each isochrone and luminosity functions of the 16 isochrones we then estimate how many stars we need to add for each individual isochrone to match the stellar population visible for WFIRST magnitude limit (Z062 $>$ 27.15: \citealt{spergel13}). These stars are then added to our mock 10 $\times$ 10 kpc WFIRST M31 halo fields with random positions within the panel, and age and metallicity corresponding to their respective isochrone.

Additionally, we convert the deep optical fields from the \citet{brown09} M31 HST halo fields (F606W $>$ 32) to WFIRST bands and apply the limiting magnitude of the WFIRST guest observer program (Z062 $>$ 27.15: \citealt{spergel13}). We then use the stellar density in the three \citet{brown09} halo fields, which are located at $R_{GC}$ = 11, 21, 35 kpc to estimate the stellar density that WFIRST will observe at the location of the 10 $\times$ 10 kpc panels ($R_{GC}$ = 15, 35, 55 kpc), using a power-law fit to the M31 stellar density profile. We use these stellar densities to check the stellar densities for our new M31 halo fields for WFIRST limits. From the power-law fit to the Brown fields stellar densities, the stellar densities that WFIRST will observe in the M31 halo, are $\rho_*  \sim 7 \times 10^5$,  $\sim7 \times 10^4$,  $\sim 2 \times 10^4$ stars/degrees$^2$ at  $R_{GC}$ = 15, 35, 55 kpc, respectively. The stellar densities that we reach at the same radii by fitting isochrones and adding fake stars are  $...$, $...$, and $...$ stars/degrees$^2$ at  $R_{GC}$ = 15, 35, 55 kpc, respectively.

Figure \ref{fig:M31_wfirst} shows the same panels as Figure \ref{fig:M31_pandas}, however the background fields now represent WFIRST limiting magnitude and stellar densities. We inject the same mock stellar streams as in Figure \ref{fig:M31_pandas}, but now we populate the streams with a larger amount of stars reflecting the limiting magnitudes and bands of WFIRST (see bottom row of Figure \ref{fig:iso_cfht}). In particular, we populate the Pal 5-like stream with 245 stars and the more massive stream with 2467 stars.

From Figure \ref{fig:M31_wfirst}, it is evident that a Pal 5-like stream will be detectable, by eye, in M31 with WFIRST if we apply metallicity cuts and $R_{GC} > 35$ kpc (see right two panels in middle row) or  $R_{GC} > 15$ kpc  (see  bottom row). This provides exciting prospects for the detection of thin, stellar streams in M31's halo and could vastly expand our sample of known streams for which we can investigate the morphology and search for gaps in streams. 




\begin{figure}
\centerline{\includegraphics[width=\columnwidth]{Fig4_distance.pdf}}
\caption{{\bf Cold streams in external galaxies: resolved stars}. The number of resolved WFIRST $Z062$-band stars in a Pal 5-like stream (solid) and a ten times more massive Pal 5-like stream (dashed) as a function of distance. We compute this Figure based on the cumulative luminosity function in the WFIRST $Z062$-band (see right, bottom panel of Figure \ref{fig:iso_cfht}). The vertical lines show the distances to four well-known galaxies within 10 Mpc of the Milky Way. The red star shows the number of stars needed for the Pal 5-like stream to have  C$_{\rm err} > 15$ and thus stand out against an M31-like background at the specific distance at a galactocentric radius of 35 kpc and after a metallicity cut of [Fe/H] $< 0$. The red ``+" similarly shows this for the more massive stream. 
Pal 5-like streams should be detectable in resolved stars with WFIRST out to the distance of Cen A (see intersect of solid line and dark grey shaded region), and ten times more massive streams should be observed out to distances slightly smaller than M101 (see intersect of dashed line and light gray shaded region). %This conclusion is drawn without the use of color-cuts, and we could also use more conservative metallicity cuts than in Figure \ref{fig:M31_pandas}. As seen in Figure \ref{fig:M31_wfirst}, this will also depend on the location in the host galaxy (easier to detect in outer halo), and the exact background and metallicity of the specific host. 
Note that beyond 11.5 Mpc no stars will be resolved with WFIRST in a Pal 5-like stream.}
\label{fig:distance}
\end{figure}


\subsection{Cold globular cluster streams in external galaxies: resolved stars}
\label{sec:resother}
In the previous Section, we showed that only some streams were detectable by eye against the background of M31. To test the threshold of when streams are detectable, we compute the contribution of the stream signal to the M31 background at different galactocentric radius ($R_{GC} = 15, 35, 55$ kpc), with three different metallicity cuts ([Fe/H] = all, $<$ 0, $< -1$). In particular, we compute the counting error, C$_{\rm err}$: 

\begin{equation}
{\rm C}_{\rm err} = \frac{N_{\rm stream}}{\sqrt{N_{\rm background}}}
\end{equation}
where $N_{\rm stream}$ is the number of stars in the mock streams, and $N_{\rm background}$ is the number of stars in the M31 background within the same covered area as the specific mock stream (see Table \ref{tab:Pal5}). 

We vary $N_{\rm stream}$ and determine, through visual inspections of mock streams injected to the various panels in Figure \ref{fig:M31_pandas} and \ref{fig:M31_wfirst}, that only streams with C$_{\rm err} > 15$ are detectable by eye against the backgrounds, at all $R_{GC}$ with the three different [Fe/H] cuts. As a reference, in Figure \ref{fig:M31_pandas} the Pal 5-like stream has C$_{\rm err} < 15$ in all 9 panels, although the bottom right panel has a  C$_{\rm err}$ value very close to 15. For the more massive stream, C$_{\rm err} > 15$ in the two right middle panels of and all lower panels of Figure \ref{fig:M31_pandas}. Note that all of these values have  C$_{\rm err} > 10$, and hints of the massive streams are seen in the other panels. 

In Figure \ref{fig:distance} we show the number of resolved stars in a Pal 5-like stream (solid) and a ten times more massive stream (dashed) as a function of distance to external galaxies. We used the normalized luminosity function shown in the bottom panel of Figure \ref{fig:iso_cfht} based on WFIRST's limiting magnitude (Z062 $<$ 27.15), to obtain the number of resolved stars various distances. The vertical lines show the distances to four well-known galaxies in which it could be interesting to search for GC stellar streams. 

Motivated by our finding that only streams that have C$_{\rm err} > 15$ are detectable by eye,  we illustrate the threshold for the number of stars needed to detect the streams. In particular, we add a red star illustrating the threshold at which C$_{\rm err} > 15$ for the Pal 5-like stream and a red ``+" for the more massive stream. The star and ``+" markers are determined by assuming the two mock streams are located at $R_{GC} = 35$ kpc within their host galaxies (to fix the areas of the two mock streams). We additionally apply a metallicity cut of [Fe/H] $<0$ to only use one background estimate.  At each distance, we estimate the number of stars needed to ensure that C$_{\rm err} > 15$ for each stream. Thus, if the solid line is above the red star, the Pal 5-like stream is observable. Similarly, if the dashed line is above the red ``+", the 10 times more massive stream is observable. 

We conclude that Pal 5-like streams should be detectable with WFIRST out to the distance of Cen A, and that ten times more massive streams should be observed out to the distance of M101. We encourage observers to look for these streams in the future, and we emphasize that the results presented in Figure \ref{fig:distance} are conservative, as we could do even better with stricter color cuts (i.e. metallicity cuts). Moreover, our estimates are ``by eye" - and could do better applying some sort of averaging (e.g. \citealt{malhan18}) (see discussion in Section \ref{sec:discussion}). It is important to note again, however, that beyond 11.5 Mpc WFIRST will not resolve any stars in a Pal 5-like stream, as the tip of the red giant branch of Pal 5 would be be below the detection limit at this distance (see isochrone in bottom left part of Figure \ref{fig:iso_cfht}). To detect thin streams at larger distances than 11.5 Mpc, the GCs would need to have been much younger than Pal 5 to host brighter stars. 

Additionally, we need to ensure that given the number densities of the streams, all stars will be resolved with WFIRST at each distance. Thus, for each mock stream, which span various areas, we calculated the number density of stars. In particular, we calculate the number of stars that can be resolved with WFIRST at a given distance and divide this by the area of each stream in arcsec$^2$ at this given distance. As the stream is moved to a greater distance, we will be lose the amount of resolved stars due to WFIRST's limiting magntidude (see Figure \ref{fig:distance}), while the area of the streams decrease. 

To test whether there will be crowding in WFIRST's pixels (i.e. more than 1 star per pixel), we compare the number density above to the pixel size of WFIRST, which is $0.11\arcsec \times 0.11\arcsec$. Hence, we multiply the number of stars per arcsec$^2$ in each stream and at each distance by the area of each WFIRST pixel in arcsec$^2$.
The area of the six different streams vary due to the change in length and width of the streams depending on their location in their host galaxies and their mass (see Table \ref{tab:Pal5}). 

We show the result of one of these calculations in Figure \ref{fig:dens}, where we plot the number of stars per WFIRST pixel for a Pal 5-like stream located at $R_{GC} = 35$ kpc (solid) and a 10 times more massive stream located at $R_{GC} = 35$ kpc (dashed). Note that the shape of the solid and dashed curve is determined by the luminosity function as we lose certain parts of the luminosity function when moving the streams to greater distances. From Figure \ref{fig:dens}, we conclude that crowding will not be an issue for resolving stars in thin stellar streams with WFIRST. 


\begin{figure}
\centerline{\includegraphics[width=\columnwidth]{stars_per_pixel.png}}
\caption{ The number of resolved stars per WFIRST pixel (Area = $0.11\arcsec \times 0.11\arcsec$) for a Pal 5-like stream (solid) and 10 times more massive stream (dashed) as a function of distance to their host galaxies. The area of both streams have been fixed based on assuming they are located 35 kpc from their hosts' galactic centers. The shape of the lines reflect the fact that we are losing certain parts of the luminosity function due to the limiting magnitude of WFIRST at each distance. Note that there is less than one star per WFIRST pixel at each distance. \todo{We can maybe omit this figure?}}
%HSC: i is the best band, median seeing (limiting resolution) in i is 0.6 arcsec (wavelength dependent), psf for HSC: we need to find. WFIRST imaging should also be on this plot (0.12-0.14 depending on band, psf full width half max.  }
\label{fig:dens}
\end{figure}




%http://www.stsci.edu/hst/wfc3/documents/handbooks/currentIHB/c06_uvis09.html


\subsection{Cold streams in external galaxies: Integrated light}
\label{sec:integrated}
We can also search for thin stellar streams using integrated light. Current telescopes such as HSC (\citealt{miyazaki12}) and future surveys as WFIRST's imaging program (\citealt{spergel13}) and LSST  (\citealt{ivezi08}) are ideal for integrated light searches for thin, stellar streams. 

To address whether we should find streams in external galaxies in integrated light, we compute the surface brightness of the Pal 5-like and ten times more massive streams at three different galactocentric radii ($R_{GC}$ = 15  kpc, $R_{GC}$ = 35 kpc and $R_{GC}$ = 55 kpc, see Table \ref{tab:Pal5}). 

We compute the surface brightness, $\mu$, by converting the magnitudes of each star in the normalized luminosity function (see middle, bottom row, Figure \ref{fig:iso_cfht}), to a luminosity. Subsequently, we sum up the luminosities for all stars in the stream to obtain a total luminosity. We divide the total luminosity by the area of each stream in arcsec$^2$ and calculate the surface brightness, $\mu$, in [mag/arcsec$^2$]. See all values in Table \ref{tab:Pal5}. The surface brightness of a stream is independent of the distance to its host galaxy, as both the total luminosity and the area of the stream decreases with distance squared. The values, however, will change depending on the telescope bands (as the total luminosity changes) and also the IMF sampler to compute the total luminosity. See the range of the mock stream widths and their surface brightnesses in the legend of Figure \ref{fig:int}. 

The width of a stream decreases with distance to the external host galaxy. To determine whether we would resolve GC stellar streams with current and upcoming telescopes, given the telescopes resolution limits, in Figure \ref{fig:int} we plot the range of widths for our six mock-streams (blue shaded region). Recall that smaller galactocentric radii and lower mass streams, yield narrower streams and vice versa. %Note that the narrowest massive stream overlaps with the second narrowest Pal 5-like stream. 

The vertical lines in Figure \ref{fig:int} show the distances to four nearby galaxies, which could be targeted for thin stellar stream observations. We have labeled the vertical lines by the typical surface brightness of the stellar halos for these systems. In particular, LMC has a surface brightness $\mu \sim 34$ mag/arcsec$^2$ beyond 20 kpc (\citealt{nidever18}), M31 has a typical surface brightness of  $\mu \sim$ 32 mag/arcsec$^2$ beyond 40 kpc (\citealt{ibata07}, fig. 42), Cen A has  $\mu \sim $ 32 mag/arcsec$^2$ (\citealt{crno16}), and M101 has  $\mu \sim 32 $ mag/arcsec$^2$ beyond 40 kpc (\citealt{dokkum14}, fig. 2).

The horizontal lines show the resolution limitations of HSC, LSST (limited by seeing: 0.7$\arcsec$) and WFIRST (limited by the pixel scale: 0.11$\arcsec$: \citealt{spergel13}).

From Figure \ref{fig:int}, we conclude that Pal 5-like streams are observable in integrated light to $\sim 2\times 10^4$ kpc with HSC and LSST, and that 10 times more massive streams should be detectable out to $ d \sim$ 10$^5$ kpc (see intersect of HCS/LSST horizontal line with upper and lower limit of blue shaded region). Similarly, WFIRST will be able to detect a Pal 5-like stream out to $\sim 10^5$ kpc, and a ten times more massive stream out to $d \sim 6 \times 10^5$ kpc (see intersect of WFIRST horizontal line with upper and lower limit of blue shaded region). However, at a distance of $d \sim 6 \times 10^5$ kpc, the length of a 10 times more massive stream than Pal 5 would only be $\sim 6 \arcsec$ (and the width $\sim 0.11\arcsec$). 

The statements about detectability above are only valid if the surface brightness of the streams supersede their hosts' halo surface brightnesses. By comparing the values of the surface brightnesses of the external galaxies with the values for the six streams in Table \ref{tab:Pal5}, we caution that this might be a difficult exercise. However, one could also apply metallicity cuts to the halo stars, which would make it easier to detect the streams (see e.g. \citealt{ibata14}).


%Have simple calculation based on survey limit, mass of GC and how many stars you should find. Easy for observers to grab for proposals. 
%Figure \ref{fig:int}: compute surface brightness for all 6 streams depending on their area (length and width). Surface brightness does not vary with distance. 




%\begin{figure}
%\centerline{\includegraphics[width=\columnwidth]{M31_Pal5.pdf}}
%\caption{This plot should have Pal 5-like stream streams at various radii where the length are updated. The streams should be 'great circles' instead of Pal 5 projections. }
%\label{fig:M31}
%\end{figure}




\begin{figure}
\centerline{\includegraphics[width=\columnwidth]{Fig5_integrated.pdf}}
\caption{{\bf Cold streams in external galaxies: integrated light.} The angular width in arcsec of a Pal 5-like stream and a ten times more massive Pal 5-like stream (spanning the range of the shaded blue region) as a function of distance to the streams' external host galaxies. The stream widths are calculated at three different galactocentric radii in a host Galaxy with a mass-profile similar to M31 ($v_{circ} = 250 ~\kms$). The stream is thinnest at smaller $R_{GC}$ and for the lowest mass cluster (see Section \ref{sec:length}). The vertical line show the distances to four well-known galaxies which are labeled with their typical halo surface brightnesses in units of mag/arcsec$^2$. The horizontal lines show the angular resolution of three different current (HSC) and future (LSST, WFIRST) surveys. See where the blue shaded region intersects the HSC/LSST and WFIRST horizontal lines for distance limits of detectability.} %We have calculate the surface brightness of each stream as determined by their integrated light and their length and widths (see Table \ref{tab:Pal5}). The surface brightness of the streams do not change with distance. Therefore, if the surface brightness is larger than the background of a given host galaxy,  Pal 5-like streams should observable to $\sim 2\times 10^4$ kpc with HSC, LSST and a ten times more massive stream should be observable to $ d \sim$ 10$^5$ kpc in integrated light.  }
%HSC: i is the best band, median seeing (limiting resolution) in i is 0.6 arcsec (wavelength dependent), psf for HSC: we need to find. WFIRST imaging should also be on this plot (0.12-0.14 depending on band, psf full width half max.  }
\label{fig:int}
\end{figure}


\section{Discussion}
\label{sec:discussion}

\subsection{Finding thin globular cluster streams}
\label{sec:color}
Throughout the paper, we have provided conservative estimates for the detectability of thin stellar streams in external galaxies. In reality, there are additional techniques we can apply which can facilitate detections of thin streams to greater distances. 

Globular clusters have metallicity spreads from [Fe/H]$ = -2.5$ - 0 (\citealt{harris96}). Depending on the properties of the stellar halo of interest, the cluster can stand out more against the background stellar halo of their host galaxy than illustrated in Figure \ref{fig:M31_wfirst} if the clusters have lower metallicities than Pal 5. PAndAS and WFIRST provide photometric metallicities, but if we obtain spectroscopic metallicities this can enhance the distinction of streams from the background, as the color in each band would provide separate information from the metallicity. Several colors are available for WFIRST and cuts in color-color diagrams can therefore help detect fainter streams (see e.g. \citealt{shipp18}). 

In addition to color and metallicity cuts, we might be able to detect more streams in external galaxies by doing orbit searches using algorithms such as ``stream-finder" (e.g. \citealt{malhan18}, \citealt{ibata19}). This might enable us to find thin stellar streams at larger distances than suggested in Figure \ref{fig:distance}. However, the lack of kinematic information could complicate these searches. Another method to improve detectability is to use a matched filter technique by applying a smoothing criteria matching the estimated width of the stream at a given distance to the external galaxy of interest. Even without kinematics, we can use the fact that streams are contiguous and search for them using algorithms such as the ``Rolling Hough Transform" (see \citealt{clark14}). 

In Section \ref{sec:lum}, we noted that WFIRST will not be able to resolve stars in a Pal 5-like stream to greater distances than 11.5 Mpc as the limiting magnitude of WFIRST would be brighter than the brightest stars in Pal 5. Therefore, our results indicate that it is unlikely that the reported ``Maybe Stream" at 20 Mpc (\citealt{abraham18}) hosts resolved stars from the remnant of a Pal 5-like globular cluster. The observations presented in \citet{abraham18}  were carried out with one HST orbit, and they reached a limiting magnitudes of AB $\sim 26.5$ (Abraham, private communication). For comparison, the tip of the red giant branch of Pal 5 would be  below the detection limit at this distance using deeper WFIRST limits of mag $<$ 27.15, as the isochrone shifts by a factor $\sim$14.7 by moving the stream from 23 kpc to 20 Mpc (see isochrone in bottom left part of Figure \ref{fig:iso_cfht}). 

It is possible that \citet{abraham18} are observing the stream in integrated light (see Figure \ref{fig:int}), however the width of the ``Maybe Stream" stream is only 0.1$\arcsec$ at 20 Mpc, which indicates the stream is only 10 pc wide. Thus assuming 10 pc is representative of the full width of the ``Maybe Stream", the stream would be of lower initial mass than Pal 5's cluster For comparison the thin MW streams, GD1 and Pal 5 range in widths from 30-60 pc (\citealt{price18}) and 40-80 pc (Bonaca et al., {\it in prep.}), repsectively. It therefore seems unlike that a low mass cluster would produce such a long, bright stream. Though the cluster could be much younger than Pal 5 and thereby host brighter stars. Deeper observations might help resolve these remaining puzzles. 

\subsection{Gaps in cold stellar streams in external galaxies}
The gravitational interactions between dark matter subhalos and stellar streams provide an intriguing method for probing the dark matter subhalo power spectrum and thereby setting limits on the nature of the dark matter particle (see e.g. \citealt{erkal16}, \citealt{bovy17}, \citealt{price18}, \citealt{bonaca19}). \citet{garrison17} showed that the presence of dark matter subhalos should be suppressed substantially within the orbit of Pal 5 if baryonic disks are included in dark matter only simulations of galaxies.  Despite this fact, gaps and irregularities have been reported in Pal 5 (see e.g. \citealt{erkal17}, Bonaca et al., {\it in prep.}). These irregularities likely arise to due to Pal 5's prograde orientation with respect to the disk of the Galaxy, as Pal 5 will more likely be affected by molecular clouds in the disk (\citealt{amorisco16}), torques from the Galactic bar (\citealt{hattori16}, \citealt{erkal17}, \citealt{pearson17}) or interactions with spiral arms (\citealt{banik19}).

%%This estimates of subhalo populations in the inner Galaxy were made without including baryonic physics, and \citet{garrison17} suggests that many of the subhalos should be destroyed by the MW's disk. 
The MW GC stellar stream, GD1 (\citealt{grillmair06}), has an orbit which probes a similar region of the Galaxy as Pal 5 (GD1's $r_{peri} \sim 14$ kpc, GD1's $r_{apo} \sim 26-29$ kpc, \citealt{koposov10}). The orientation of its orbit, on the other hand, is retrograde with respect to the disk of the Galaxy and its pericentric distance is larger than Pal 5's. This makes GD1 a cleaner laboratory for searching for potential past interactions with dark matter subhalos. Interestingly, the recent detection of an under density and a ``spur" in GD1 using data from Gaia DR2 (\citealt{gaiadr2}) can be interpreted as an interaction with a dense substructure (\citealt{price18}, \citealt{bonaca19})

Our work shows that WFIRST should detect many thin stellar streams in resolved stars in nearby galaxies. This will open up the possibility of exciting indirect detections of dark matter through density distortions and gaps in thin stellar streams. 

In external galaxies, we have the privilege of selecting galaxies without spiral arms and bars, limiting the possible stream perturbers. Ideal hosts could be dwarfs as well as elliptical galaxies, and we could in principle build up statistical samples to measure subhalo properties as a function of host mass. 

Furthermore, we will  have the opportunity to look for streams at greater galactocentric distances, where more subahlos should reside (\citealt{garrison17}). We will still have to exclude interactions with globular clusters and satellites (see e.g. \citealt{bonaca19}), which will be more difficult due to the lack of kinematic information. However, we hypothesize that vastly increasing the number of thin stellar streams with resolved stars in external galaxies, will be crucial in our quest for an understanding of dark matter. 

\section{Conclusion}\label{sec:conclusion}
In this paper, we have created mock globular cluster stellar streams and investigated the observability of thin, globular cluster stellar streams in external galaxies. Based on our findings we make the following conclusions:

\begin{itemize}
\item[1.]  WFIRST will be able to detect GC stellar streams in resolved stars in galaxies within xx Mpc.
\item[2.]  Current and future imaging surveys, using integrated light, will be able to detect thin globular cluster streams out to distances of $d \sim 600$ Mpc, depending on the streams' width, their location in their host galaxies, and the surface brightnesses of the hosts' halos. 
\end{itemize}


Our work provides a positive outlook on the future prospects of using thin streams in external galaxies to both indirectly probe the nature of dark matter through gaps in streams, and to map orbit structures in external galaxies (see Yavetz et al., {\it in prep.}). 

\acknowledgements
We thank Robyn Sanderson, David Spergel, Rubab Khan, Benjamin Williams and Anil C. Seth for insightful discussions on WIFRST. SP thanks Adrian Price-Whelan and David W. Hogg for insightful questions. KVJ's contributions were supported by NSF grant AST-1614743. The Flatiron Institute is supported by the Simons Foundation. This research made use of Astropy,\footnote{http://www.astropy.org} a community-developed core Python package for Astronomy \citep{astropy:2013, astropy:2018}.

\appendix\label{sec:appendix}
Throughout the paper, we have ...



\software{
    \package{Astropy} \citep{astropy:2013, astropy:2018},
    \package{matplotlib} \citep{Hunter:2007},
    \package{numpy} \citep{walt2011},
}

\bibliographystyle{aasjournal}
\bibliography{pal5ext}

\end{document}
